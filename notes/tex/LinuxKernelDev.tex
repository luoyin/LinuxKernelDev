\documentclass[10pt]{book}
\usepackage{amsmath}
\usepackage{amssymb}
\usepackage{latexsym}
\usepackage{ctex}
\usepackage{mathdots}
\usepackage{mathrsfs}
\usepackage{longtable}
\usepackage{supertabular}
\usepackage{multirow}
\usepackage{multicol}
\usepackage{array}
\usepackage{color, framed}
\usepackage{xcolor}
\usepackage{float}
\usepackage{listings}
\usepackage{threeparttable}
\usepackage{graphicx}
\usepackage{graphics}
\usepackage{enumerate}
\usepackage{tikz}
\usetikzlibrary{shapes.geometric}
\usepackage{lscape}
\usepackage[colorlinks]{hyperref}
\usepackage{geometry}
\setcounter{secnumdepth}{4}
\geometry{left=2.5cm, right=2.5cm, top=2.5cm, bottom=2.5cm}
\begin{document}

\lstset{numbers=left,
        numberstyle= \tiny,
        keywordstyle= \color{ blue!70},
        commentstyle=\color{red!50!green!50!blue!50},
        frame=shadowbox,
        rulesepcolor= \color{ red!20!green!20!blue!20},
        xleftmargin=-1em,
        xrightmargin=-1em,
        tabsize=4,
        breaklines=true,
    basicstyle=\small
}

\definecolor{shadecolor}{rgb}{0.92, 0.92, 0.92}

\newcommand{\red}[1]{\textcolor[rgb]{1.0, 0.0, 0.0}{#1}}
\newcommand{\green}[1]{\textcolor[rgb]{0.0, 1.0, 0.0}{#1}}
\newcommand{\blue}[1]{\textcolor[rgb]{0.0, 0.0, 1.0}{#1}}

\title{Linux内核详解}
\author{罗胤}
\date{2017-02-07}
\maketitle

\tableofcontents

\chapter{Linux头文件}



\end{document}